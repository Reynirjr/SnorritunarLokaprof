\documentclass{article}

% Encodings, page setup, paragraph formatting, font
\usepackage[top=0.9in, bottom=1in, left=1.5in, right=1.5in]{geometry}
\usepackage[icelandic]{babel}
\usepackage[T1]{fontenc}
\usepackage[sc]{mathpazo}
\usepackage[parfill]{parskip}
\usepackage{cancel}
% Tables and lists
\usepackage{booktabs,tabularx}
\usepackage{multirow}
\usepackage{enumerate}
\usepackage{adjustbox}
\usepackage{multicol}
\usepackage{enumitem}
\usepackage{xcolor}
% Math
\usepackage{amsmath, amsfonts, amssymb, amsthm}
% Graphics
\usepackage{graphicx}
\usepackage{forest}
\usepackage{tikz}
\usetikzlibrary{positioning, shapes, arrows.meta}
% Custom Commands til að auðvelda lífið
\newcommand{\sv}{\textbf{Svar:}}
\newcommand{\bo}[1]{\textbf{#1}}
\newcommand{\enum}{\begin{enumerate}[label = \alph*.]}


\usepackage{listingsutf8}
\definecolor{commentcolor}{RGB}{255, 0, 255} % Bleikur
\definecolor{keywordcolor}{RGB}{139, 69, 19}   % Brúnn
\definecolor{stringcolor}{RGB}{0, 0, 255}      % Blár
\definecolor{numbercolor}{RGB}{0, 128, 0}      % Grænn
\definecolor{identifiercolor}{RGB}{0, 0, 255}
%Morpho
\lstdefinelanguage{CustomLang}{
    alsoletter={=},
    keywords={rec, fun, if, else, return},
    sensitive=true,
    comment=[l]{;;;},
    commentstyle=\color{commentcolor},
    morestring=[b]",
    stringstyle=\color{stringcolor},
}

\lstset{
    language=CustomLang,
    basicstyle=\ttfamily,
    keywordstyle=\color{keywordcolor},
    commentstyle=\color{commentcolor},
    identifierstyle=\color{identifiercolor},
    stringstyle=\color{stringcolor},   
    showstringspaces=false,
    numbers=none,
    tabsize=2,
    breaklines=true,
    columns=fullflexible,
    keepspaces=true,
    inputencoding=utf8, 
    extendedchars=true,  
    literate=
        {á}{{\'a}}1
        {ð}{{\dh}}1
        {é}{{\'e}}1
        {í}{{\'i}}1
        {ó}{{\'o}}1
        {ú}{{\'u}}1
        {ý}{{\'y}}1
        {þ}{{\th}}1
        {æ}{{\ae}}1
        {ö}{{\"o}}1
        {Á}{{\'A}}1
        {Ð}{{\DH}}1
        {É}{{\'E}}1
        {Í}{{\'I}}1
        {Ó}{{\'O}}1
        {Ú}{{\'U}}1
        {Ý}{{\'Y}}1
        {Þ}{{\TH}}1
        {Æ}{{\AE}}1
        {Ö}{{\"O}}1,
}


% Hyphenation
\hyphenpenalty=5000
% Page and section numbering
\setcounter{secnumdepth}{-1} 
\pagenumbering{gobble}
\title{Forritunarmál}
\author{Benjamín}
\date{11.27.2024}

\begin{document}

\maketitle

\begin{center}
    \Huge{Lokapróf Forritunarmál 2018-2023}
\end{center}

\newpage

\section{Lokanir}

\bo{Hverjar eftirfarandi fullyrðinga um lokanir eru sannar? Tvö röng
svör gefa núll punkta.}


\enum
\item Lokanir eru til í C.
\item Lokanir eru til í Scheme.
\item Lokanir eru til í CAML.
\item Lokanir eru til í Morpho.
\item Lokanir innihalda fallsbendi.
\item Lokanir eru nauðsynlegar til að skila staðværu falli sem skilagildi
      falls í bálkmótuðum forritunarmálum.
\item Lokanir eru aðeins mögulegar ef vakningarfærslur eru í kös.
\item Lokanir innihalda stýrihlekk
\item Lokanir innihalda tengihlekk.
\item Lokanir innihalda straum.
\item Lokanir eru nauðsynlegar til að senda staðvær föll sem viðföng í bálkmótuðum forritunarmálum.
\item Lokanir má nota til að útfæra strauma í scheme.

\end{enumerate}

\begin{tabularx}{\textwidth}{|X|X|X|X|X|X|X|X|X|X|X|X|}
    \hline
    \bo{a} & \bo{b} & \bo{c} & \bo{d} & \bo{e} & \bo{f} & \bo{g} & \bo{h} & \bo{j} & \bo{k} & \bo{l} & \bo{m} \\ \hline
     & & & & & & & & & & & \\ \hline
\end{tabularx}

\newpage

\bo{Hverjar eftirfarandi fullyrðinga um aðgangshlekki (tengihlekki),
stýrihlekki og lokanir eru sannar? Tvö röng svör gefa núll punkta.}

\enum
\item Aðgangshlekkir eru notaðir í bæði bálkmótuðum og öðrum
forritunarmálum.
\item Stýrihlekkir eru notaðir bæði í bálkmótuðum og öðrum
forritunarmálum
\item Lokanir innihalda aðgangshlekk
\item Lokanir innihalda stýrihlekk og aðgangshlekk.
\item Lokanir innihalda vendivistfang og stýrihlekk.
\item Lokanir innihalda fallsbendi.
\item Lokanir innihalda fallsbendi og aðgangshlekk.
\item aðgangshlekkir eru ekki til í Haskell
\item Lokanir eru ekki til í Scheme
\item Stýrihlekkir eru ekki til í CAML.
\item Lokanir eru ekki til í Morpho.
\item Aðgangshlekkir eru ekki til í Java
\item Lokanir eru aðeins mögulegar ef vakningarfærslur eru í kös
\end{enumerate}

\begin{tabularx}{\textwidth}{|X|X|X|X|X|X|X|X|X|X|X|X|}
    \hline
    \bo{a} & \bo{b} & \bo{c} & \bo{d} & \bo{e} & \bo{f} & \bo{g} & \bo{h} & \bo{j} & \bo{k} & \bo{l} & \bo{m}  \\ \hline
     & & & & & & & & & & &  \\ \hline
\end{tabularx}

\newpage

\bo{Íhugið eftirfarandi forritstexta í Scheme.}

\begin{verbatim}
(define (p x)
...(define a 1)
    (define b (lambda () (+ a x)))
    (define c (b))
    (define (d y) (+ x y))
    (define (e z) (lambda (v) (+ x z v)))
    (define f (lambda (w) (e w)))
    (define g (f x))
    ...
)
\end{verbatim}

\enum
\item Hverjar af breytunum a,…,g inni í p innihalda lokanir?
\item Fyrir sérhverja breytu sem inniheldur lokun, tilgreinið nafn
      þess falls sem tengihlekkur lokunarinnar vísar á
      vakningarfærslu fyrir.
\end{enumerate}

\newpage
\section{Vakningarfærsla}

\bo{Vakningarfærsla falls í bálkmótuðu forritunarmáli eins og Scheme
inniheldur sum eftirfarandi atriða. Hver? Tvö röng svör gefa núll
stig}

\enum
\item Staðværar breytur fallsins
\item Bendi á vakningarfærslu fallsins sem kallaði á fallið
\item Bendi á vakningarfærslu fallsins sem inniheldur fallið, textalega
séð, ef eitthvert er
\item Skráakerfi tölvunnar
\item Viðföng fallsins
\item Aðgangshlekk (tengihlekk)
\item Stýrihlekk
\item Vendivistfang.
\item Benda á öll föll sem hægt er að kalla á úr fallinu
\item Benda á allar lifandi vakningarfærslur
\item Alla hluti sem til eru í kerfinu.
\item Vakningarfærslur allra falla sem hægt er að kalla á
\item Nöfn allra falla sem hægt er að kalla á
\item Lokun sem vísar á fallið.
\end{enumerate}


\begin{tabularx}{\textwidth}{|X|X|X|X|X|X|X|X|X|X|X|X|X|}
    \hline
    \bo{a} & \bo{b} & \bo{c} & \bo{d} & \bo{e} & \bo{f} & \bo{g} & \bo{h} & \bo{j} & \bo{k} & \bo{l} & \bo{m} & \bo{n} \\ \hline
     & & & & & & & & & & & & \\ \hline
\end{tabularx}







\end{document}