\documentclass{article}

% Encodings, page setup, paragraph formatting, font
\usepackage[top=0.9in, bottom=1in, left=1.5in, right=1.5in]{geometry}
\usepackage[icelandic]{babel}
\usepackage[T1]{fontenc}
\usepackage[sc]{mathpazo}
\usepackage[parfill]{parskip}
\usepackage{cancel}
% Tables and lists
\usepackage{booktabs,tabularx}
\usepackage{multirow}
\usepackage{enumerate}
\usepackage{adjustbox}
\usepackage{multicol}
\usepackage{enumitem}
\usepackage{xcolor}
% Math
\usepackage{amsmath, amsfonts, amssymb, amsthm}
% Graphics
\usepackage{graphicx}
\usepackage{forest}
\usepackage{tikz}
\usetikzlibrary{positioning, shapes, arrows.meta}
% Custom Commands til að auðvelda lífið
\newcommand{\sv}{\textbf{Svar:}}
\newcommand{\bo}[1]{\textbf{#1}}
\newcommand{\enum}{\begin{enumerate}[label = \alph*.]}


\usepackage{listingsutf8}
\definecolor{commentcolor}{RGB}{255, 0, 255} % Bleikur
\definecolor{keywordcolor}{RGB}{139, 69, 19}   % Brúnn
\definecolor{stringcolor}{RGB}{0, 0, 255}      % Blár
\definecolor{numbercolor}{RGB}{0, 128, 0}      % Grænn
\definecolor{identifiercolor}{RGB}{0, 0, 255}
%Morpho
\lstdefinelanguage{CustomLang}{
    alsoletter={=},
    keywords={rec, fun, if, else, return},
    sensitive=true,
    comment=[l]{;;;},
    commentstyle=\color{commentcolor},
    morestring=[b]",
    stringstyle=\color{stringcolor},
}

\lstset{
    language=CustomLang,
    basicstyle=\ttfamily,
    keywordstyle=\color{keywordcolor},
    commentstyle=\color{commentcolor},
    identifierstyle=\color{identifiercolor},
    stringstyle=\color{stringcolor},   
    showstringspaces=false,
    numbers=none,
    tabsize=2,
    breaklines=true,
    columns=fullflexible,
    keepspaces=true,
    inputencoding=utf8, 
    extendedchars=true,  
    literate=
        {á}{{\'a}}1
        {ð}{{\dh}}1
        {é}{{\'e}}1
        {í}{{\'i}}1
        {ó}{{\'o}}1
        {ú}{{\'u}}1
        {ý}{{\'y}}1
        {þ}{{\th}}1
        {æ}{{\ae}}1
        {ö}{{\"o}}1
        {Á}{{\'A}}1
        {Ð}{{\DH}}1
        {É}{{\'E}}1
        {Í}{{\'I}}1
        {Ó}{{\'O}}1
        {Ú}{{\'U}}1
        {Ý}{{\'Y}}1
        {Þ}{{\TH}}1
        {Æ}{{\AE}}1
        {Ö}{{\"O}}1,
}


% Hyphenation
\hyphenpenalty=5000
% Page and section numbering
\setcounter{secnumdepth}{-1} 
\pagenumbering{gobble}
\title{Forritunarmál}
\author{Benjamín}
\date{11.27.2024}

\begin{document}

\maketitle

\begin{center}
    \Huge{Lokapróf Forritunarmál 2018, 2019, 2021, 2022,2023}

    \bo{Í þessu samansafni tek ég ekki inn 2020 þar sem það er svo frábrugðið hinum prófunum,
    einnig sleppi ég stundum 2021 þar sem það er einnig frábrugðið og var covid próf}
\end{center}



\newpage

\section{Lokanir}

\bo{Hverjar eftirfarandi fullyrðinga um lokanir eru sannar? Tvö röng
svör gefa núll punkta.}


\enum
\item Lokanir eru til í C.
\item Lokanir eru til í Scheme.
\item Lokanir eru til í CAML.
\item Lokanir eru til í Morpho.
\item Lokanir innihalda fallsbendi.
\item Lokanir eru nauðsynlegar til að skila staðværu falli sem skilagildi
      falls í bálkmótuðum forritunarmálum.
\item Lokanir eru aðeins mögulegar ef vakningarfærslur eru í kös.
\item Lokanir innihalda stýrihlekk
\item Lokanir innihalda tengihlekk.
\item Lokanir innihalda straum.
\item Lokanir eru nauðsynlegar til að senda staðvær föll sem viðföng í bálkmótuðum forritunarmálum.
\item Lokanir má nota til að útfæra strauma í scheme.

\end{enumerate}

\begin{tabularx}{\textwidth}{|X|X|X|X|X|X|X|X|X|X|X|X|}
    \hline
    \bo{a} & \bo{b} & \bo{c} & \bo{d} & \bo{e} & \bo{f} & \bo{g} & \bo{h} & \bo{j} & \bo{k} & \bo{l} & \bo{m} \\ \hline
     & & & & & & & & & & & \\ \hline
\end{tabularx}

\newpage

\bo{Hverjar eftirfarandi fullyrðinga um aðgangshlekki (tengihlekki),
stýrihlekki og lokanir eru sannar? Tvö röng svör gefa núll punkta.}

\enum
\item Aðgangshlekkir eru notaðir í bæði bálkmótuðum og öðrum
forritunarmálum.
\item Stýrihlekkir eru notaðir bæði í bálkmótuðum og öðrum
forritunarmálum
\item Lokanir innihalda aðgangshlekk
\item Lokanir innihalda stýrihlekk og aðgangshlekk.
\item Lokanir innihalda vendivistfang og stýrihlekk.
\item Lokanir innihalda fallsbendi.
\item Lokanir innihalda fallsbendi og aðgangshlekk.
\item aðgangshlekkir eru ekki til í Haskell
\item Lokanir eru ekki til í Scheme
\item Stýrihlekkir eru ekki til í CAML.
\item Lokanir eru ekki til í Morpho.
\item Aðgangshlekkir eru ekki til í Java
\item Lokanir eru aðeins mögulegar ef vakningarfærslur eru í kös
\end{enumerate}

\begin{tabularx}{\textwidth}{|X|X|X|X|X|X|X|X|X|X|X|X|}
    \hline
    \bo{a} & \bo{b} & \bo{c} & \bo{d} & \bo{e} & \bo{f} & \bo{g} & \bo{h} & \bo{j} & \bo{k} & \bo{l} & \bo{m}  \\ \hline
     & & & & & & & & & & &  \\ \hline
\end{tabularx}


\newpage
\section{Vakningarfærsla}

\bo{Vakningarfærsla falls í bálkmótuðu forritunarmáli eins og Scheme
inniheldur sum eftirfarandi atriða. Hver? Tvö röng svör gefa núll
stig}

\enum
\item Staðværar breytur fallsins
\item Bendi á vakningarfærslu fallsins sem kallaði á fallið
\item Bendi á vakningarfærslu fallsins sem inniheldur fallið, textalega
séð, ef eitthvert er
\item Skráakerfi tölvunnar
\item Viðföng fallsins
\item Aðgangshlekk (tengihlekk)
\item Stýrihlekk
\item Vendivistfang.
\item Benda á öll föll sem hægt er að kalla á úr fallinu
\item Benda á allar lifandi vakningarfærslur
\item Alla hluti sem til eru í kerfinu.
\item Vakningarfærslur allra falla sem hægt er að kalla á
\item Nöfn allra falla sem hægt er að kalla á
\item Lokun sem vísar á fallið.
\end{enumerate}


\begin{tabularx}{\textwidth}{|X|X|X|X|X|X|X|X|X|X|X|X|X|}
    \hline
    \bo{a} & \bo{b} & \bo{c} & \bo{d} & \bo{e} & \bo{f} & \bo{g} & \bo{h} & \bo{j} & \bo{k} & \bo{l} & \bo{m} & \bo{n} \\ \hline
     & & & & & & & & & & & & \\ \hline
\end{tabularx}

\newpage

\section{Foldun}

Íhugið mydnina sem sýnir földun A,B,C,D og E

\includegraphics[scale = 1 ]{myndir/AbcdFoldun.png}

Samsvarandi Scheme forritstexti er einnig sýndur í tveimur
jafngildum útgáfum hlið við hlið.

\begin{verbatim}
    (define (A ...)              (define (A ...)
     (define (B ...)              (define (E ...)
      (define (C ...)               ...[stofn E/body of E]
       (define (D ...)            )
        ...[stofn D/body of D]    (define (B ...)
       )                           (define (C ...)
       ...[stofn C/body of C]       (define (D ...)
      )                                ...[stofn D/body of D]
      ...[stofn B/body of B]        )
     )                              ...[stofn C/body of C]
     (define (E ...)               ) 
       ...[stofn E/body of E]      ...[stofn B/body of B]
     )                            )
     ...[stofn A/body of A]       ...[stofn A/body of A]
    )                            )
\end{verbatim}

Fyllið út eftirfarandi töflur með því að setja krossa við sannar fullyrðingar.
Eitt rangt svar gefur núll í einkunn fyrir dæmið.

kalla má A úr:


\begin{tabularx}{\textwidth}{|X|X|X|X|X|}
    \hline
    \bo{A} & \bo{B} & \bo{C} & \bo{D} & \bo{E}\\ \hline
    & & & & \\ \hline
\end{tabularx}

\newpage
Kalla má B úr:


\begin{tabularx}{\textwidth}{|X|X|X|X|X|}
    \hline
    \bo{A} & \bo{B} & \bo{C} & \bo{D} & \bo{E}\\ \hline
    & & & & \\ \hline
\end{tabularx}


\vspace{1cm}

Kalla má C úr:


\begin{tabularx}{\textwidth}{|X|X|X|X|X|}
    \hline
    \bo{A} & \bo{B} & \bo{C} & \bo{D} & \bo{E}\\ \hline
    & & & & \\ \hline
\end{tabularx}

\vspace{1cm}

Kalla má D úr: 


\begin{tabularx}{\textwidth}{|X|X|X|X|X|}
    \hline
    \bo{A} & \bo{B} & \bo{C} & \bo{D} & \bo{E}\\ \hline
    & & & & \\ \hline
\end{tabularx}


\vspace{1cm}

Kalla má E úr:


\begin{tabularx}{\textwidth}{|X|X|X|X|X|}
    \hline
    \bo{A} & \bo{B} & \bo{C} & \bo{D} & \bo{E}\\ \hline
    & & & & \\ \hline
\end{tabularx}

\vspace{1cm}

Staðværar breytur í A má nota í:


\begin{tabularx}{\textwidth}{|X|X|X|X|X|}
    \hline
    \bo{A} & \bo{B} & \bo{C} & \bo{D} & \bo{E}\\ \hline
    & & & & \\ \hline
\end{tabularx}

\vspace{1cm}

Staðværar breytur í B má nota í:


\begin{tabularx}{\textwidth}{|X|X|X|X|X|}
    \hline
    \bo{A} & \bo{B} & \bo{C} & \bo{D} & \bo{E}\\ \hline
    & & & & \\ \hline
\end{tabularx}

\vspace{1cm}

Staðværar breytur í C má nota í:


\begin{tabularx}{\textwidth}{|X|X|X|X|X|}
    \hline
    \bo{A} & \bo{B} & \bo{C} & \bo{D} & \bo{E}\\ \hline
    & & & & \\ \hline
\end{tabularx}

\vspace{1cm}

Staðværar breytur í D má nota í:


\begin{tabularx}{\textwidth}{|X|X|X|X|X|}
    \hline
    \bo{A} & \bo{B} & \bo{C} & \bo{D} & \bo{E}\\ \hline
    & & & & \\ \hline
\end{tabularx}

\vspace{1cm}

Staðværar breytur í E má nota í:


\begin{tabularx}{\textwidth}{|X|X|X|X|X|}
    \hline
    \bo{A} & \bo{B} & \bo{C} & \bo{D} & \bo{E}\\ \hline
    & & & & \\ \hline
\end{tabularx}


\newpage
\section{eitthver forritstexti}
Eftirfarandi forritstexti er í einhverju ímynduðu forritunarmáli.

\begin{verbatim}
    void f(x,y)
    {
        y = 3;
        print x,y;
        x = 2;
    }
    int i,a[10];
    for( i=0 ; i!=10 ; i++ ) a[i]=i+1;
    f(a[a[0]],a[0]);
    print a[0], a[1], a[2], a[3];
\end{verbatim}
Hvað skrifar þetta forrit (sex gildi í hvert skipti) ef viðföngin eru:

\enum
\item \bo{Gildisviðföng }
\item \bo{Tilvísunarviðföng }
\item \bo{Nafnviðföng}
\end{enumerate}

\newpage
\section{Hluti II – Listavinnsla o.fl.}

\bo{spurning 5 möguleikar}
\begin{enumerate}
    \item Skrifið fall í Scheme, CAML, Morpho eða Haskell sem tekur eitt
    viðfang sem er listi lista af fleytitölum milli 0 og 1 og skilar tölu
    sem er stærsta lággildi innri listanna, þ.e. stærst af þeim tölum
    sem fást þegar fundin er minnsta tala í hverjum innri lista. Þið
    skuluð reikna með því að hágildi í tóma menginu sé 0 og lággildi
    í tóma menginu sé 1. Munið fallslýsingar, eins og alltaf. Fallið
    þarf að skila viðeigandi gildi bæði fyrir tóman lista og fyrir lista
    sem einungis inniheldur tóma lista.

    \item Skrifið halaendurkvæmt fall í Scheme, CAML, Morpho eða Haskell,
    sem tekur lista talna $x1, \ldots , xn$ sem viðfang og skilar summunni
    $\sum_{i=1}^{n}X_i^2$. Þið munið þurfa hjálparfall og munið að skrifa réttar
    notkunarlýsingar. Einungis má nota einföld innbyggð föll svo sem
    +, *, null? car, cdr og cons, en ekki flóknari föll svo sem foldl eða
    map.

\end{enumerate}

\bo{spurning 6  möguleikar}
\begin{enumerate}
    \item Skrifið fall zip2 í Scheme, CAML, Morpho eða Haskell sem tekur
    tvíundaraðgerð (fall) og tvo jafnlanga lista sem viðföng og skilar
    lista þeirra útkomna sem fást þegar tvíundaraðgerðinni er beitt á
    gildin í listunum, par fyrir par. Til dæmis, í Scheme þá ætti
    segðin (zip2 + ‘(1 2 3) ‘(4 5 6)) að skila listanum (5 7 9). Notið
    einungis einfaldar aðgerðir svo sem car, cdr, cons, null?

    \item Skrifið halaendurkæmt fall zipMapRev í Scheme, CAML,
    Morpho eða Haskell sem tekur tvö viðföng sem eru jafnlangir
    listar. Fyrra viðfangið skal vera listi einundarfalla, $f1, \ldots , fn,$ og
    seinna viðfangið skal vera listi gilda $x1, \ldots , xn$ þannig að sérhvert
    $xi$ er löglegt viðfang í samsvarandi $fi$. Fallið skal skila
    viðsnúnum lista gildanna sem föllin skila þegar þeim er beitt á
    gildin, þ.e. lista með gildunum $fn(xn), ... , f1(x1)$, í þeirri röð. Notið
    einungis einfaldar aðgerðir svo sem car, cdr, cons, null?. Í
    Morpho má nota lykkju, með fastarðingu lykkju
\end{enumerate}

\bo{spurning 7 möguleikar}
\begin{enumerate}
    \item Skrifið ykkar eigin útgáfur af föllunum tveimur sem í CAML Light
    eru kölluð \text{it\_list og list\_it}. Í Haskell eru þau kölluð foldl og foldr. Þið
    megið skrifa þessi föll í Scheme, CAML, Morpho eða Haskell. Notið
    ekki lykkjur í Morpho. Kallið föllin myLeft og myRight. Þið megið
    nota aðra röð viðfanga en í \text{it\_list og list\_it}. Sjáið til þess að a.m.k.
    annað fallið sé halaendurkvæmt og tiltakið hvort það er. Notið
    aðeins einföld innbyggð föll svo sem car, cdr og null?.

    \item Skrifið tvö föll, findFirst og findLast í Scheme, CAML, Morpho
    eða Haskell, sem bæði taka eitt viðfang sem skal vera listi
    heiltalna. Skilagildið úr findFirst skal vera lengsti undirlisti
    viðfangsins sem hefur null í hausnum. Ef ekkert núll er í
    viðfanginu skal skila tómum lista. Skilagildið úr findLast skal
    vera stysti undirlisti viðfangsins sem hefur null í hausnum. Ef
    viðfangið inniheldur ekkert null skal skila tómum lista. Í þessu
    dæmi reiknum við með að undirlisti lista sé listi sem fenginn er
    með því að fjarlægja hausinn null sinnum eða oftar
\end{enumerate}
\newpage

\bo{Spurning 8 möguleikar}
\begin{enumerate}
    \item Skrifið fall mapreduce í Scheme, CAML, Morpho eða Haskell
    þannig að (í CAML) segðin mapreduce op f x u sé jafngilt segðinni
    \text{list\_it} op (map f x) u. Notið aðeins einfaldar innbyggðar aðgerðir í
    lausninni, so sem hd, tl, :: og ==. Ekki má nota map eða \text{list\_it}.
    Munið að \text{list\_it} reiknar frá hægri til vinstri.

    \item Skrifið fall revmap í Scheme, CAML, Morpho eða Haskell
    þannig að (í CAML) segðin revmap f x sé jafngild segðinni
    rev(map f x), þar sem rev er fallið sem snýr við lista. Notið
    aðeins einfaldar innbyggðar aðgerðir í lausninni, so sem hd, tl, ::
    og ==. Ekki má nota map eða \text{it\_list eða list\_it} eða rev eða önnur
    flókin föll. Þið megið skrifa og nota hjálparföll, sem þurfa þá að
    sjálfsögðu lýsingu eins og revmap þarf einnig.

    \item   Skrifið halaendurkvæmt fall í Scheme, CAML, MORPHO eða Haskell
    sem tekur sem viðföng einn lista talna, x, auk tveggja talna $a$ og $b$,
    og skilar lista þeirra talna $z$ innan $x$ þar sem $a \leq z \leq$. þið munuð 
    vilja nota hjálparfall.
\end{enumerate}

\newpage
\section{Hluti III – Einingaforritun o.fl.}

\bo{Spurning 9 möguleikar}
\begin{enumerate}
    \item Útfærið, að hluta, einingu fyrir fjölnota forgangsbiðröð í Morpho.
    Sýnið eftirfarandi.
        \bo{a.} Hönnunarskjal sem inniheldur lýsingar (notkun/fyrir/eftir) fyrir
    öll innflutt og útflutt atriði einingarinnar.
        \bo{b.} Smíð einingarinnar, þar sem sleppa má útfærslu allra
    aðgerða nema þeirri sem fjarlægir gildi úr forgangsbiðröðina.
    Athugið að sýna þarf fastayrðingu gagna.
    Unnt skal vera að nota einingaraðgerðir til að búa til afbrigði af
    einingunni sem gefa forgangsbiðraðir fyrir hvaða gildi sem er sem
    hafa viðeigandi samanburðarfall. Þið ráðið hvort forgangsbiðröðin er
    útfærð sem hlutur eða ekki.

    \item Útfærið, að hluta, fjölnota einingu fyrir poka samanburðarhæfra
    gilda í Morpho eða Java. Sýnið eftirfarandi.
    Implement, partially, a module for a bag of comparable values in
    Morpho or Java. Show the following.
        \bo{a.} Hönnunarskjal sem inniheldur lýsingar (notkun/fyrir/eftir) fyrir
        öll innflutt og útflutt atriði einingarinnar.
        \bo{b} Smíð einingarinnar, þar sem sleppa má útfærslu allra
        aðgerða nema þeirri sem fjarlægir minnsta gildi úr pokanum.
        Athugið að sýna þarf fastayrðingu gagna.
\end{enumerate}

\vspace{2cm}

\bo{Spurning 10 Möguleikar}
Hverjar af eftirfarandi fullyrðingum eru í samræmi við meginregluna
um upplýsingahuld? Það gætu verið núll, ein eða fleiri. Tvö röng
svör gefa núll stig.

\enum
\item Notendur einingar geta breytt fastayrðingu gagna
einingarinnar.
\item Gefa skal notendum einingar fullkomnar upplýsingar um smíð
einingarinnar.
\item Notendur einingar eiga að vita hver fastayrðing gagna er fyrir
eininguna.
\item Fastayrðing gagna einingar skal halda leyndri fyrir smiðum
einingarinnar.
\item Fastayrðing gagna skal vera hluti af opinberu hönnunarskjali
einingarinnar.
\item Fastayrðing gagna einingar skal ekki vera aðgengileg
notendum einingarinnar.
\item Smiðir einingar geta breytt fastayrðingu gagna einingarinnar.
\item Tilgangur upplýsingahuldar er að verja iðnaðarleyndarmál.
\item Tilgangur upplýsingahuldar er að auðvelda viðhald.
\item Fastayrðing gagna einingar skal ekki vera aðgengileg
notendum einingarinnar.

\end{enumerate}

\begin{tabularx}{\textwidth}{|X|X|X|X|X|X|X|X|X|}
    \hline
    \bo{a} & \bo{b} & \bo{c} & \bo{d} & \bo{e} & \bo{f} & \bo{g} & \bo{h} & \bo{j} \\ \hline
     & & & & & & & & \\ \hline
\end{tabularx}



\newpage

\bo{Spurning 11 Möguleikar}

    Íhugið klasa A og B þar sem B er undirklasi A. Gerið ráð fyrir að
    klasi A innihaldi boð f með eftirfarandi lýsingu.


    // Notkun: $z = t.f(x);$


    // Fyrir: $0.5 \leq x \leq 1.5$ (version 2 er $0.5 \leq x \leq 2.0$ )


    // Eftir: $|z - \sqrt{x}| < 0.01$ (version 2 er $|z^2 - x| < 0.01$ )


    Gerið ráð fyrir að í klasa B sé boðið f endurskilgreint með


    // Notkun: $z = t.f(x);$


    // Fyrir: $F_B$


    // Eftir: $E_B$


    Hverjir af eftirfarandi möguleikum fyrir $F_B$ og $E_B$ væru þá í lagi? Eitt
    rangt svar gefur núll fyrir dæmið.

    \begin{tabularx}{\textwidth}{|X|X|}
        \hline
        $F_B$ & \bo{Í lagi} \\ \hline
        $0.2 \leq x \leq 1.5 $ & \\ \hline
        $0.4 < x \leq 1.4 $ & \\ \hline
        $0.1 \leq x \leq 5.5 $ & \\ \hline
        $0.9 \leq x \leq 1.5 $& \\ \hline
        
    \end{tabularx}

    \begin{tabularx}{\textwidth}{|X|X|}
        \hline
        $E_B$ & \bo{Í Lagi} \\ \hline
        $|z^2 - x| < 0.1 $ & \\ \hline
        $|z^2 - x| \leq 0.005 $ & \\ \hline
        $|z^2 - x| < 0.02 $ & \\ \hline
        
    \end{tabularx}


    \bo{Version 2}
    \begin{tabularx}{\textwidth}{|X|X|}
        \hline
        $F_B$ & \bo{Í lagi} \\ \hline
        $0.1 \leq x \leq 5.5 $ & \\ \hline
        $0.2 \leq x \leq 1.5 $ & \\ \hline
        $0.4 < x \leq 1.4 $ & \\ \hline
        $0.9 \leq x \leq 1.5 $& \\ \hline
        
    \end{tabularx}
    \begin{tabularx}{\textwidth}{|X|X|}
        \hline
        $E_B$ & \bo{Í Lagi} \\ \hline
        $|z^2 - x| \leq 0.005 $ & \\ \hline
        $|z^2 - x| \leq 0.1 $ & \\ \hline
        $|z^2 - x| < 0.02 $ & \\ \hline
        
    \end{tabularx}

    \newpage
    \section{Hluti IV – Ýmislegt}

    \bo{Spurning 12 möguleikar:}
    \begin{enumerate}
        \item Lýsið ruslasöfnunaraðferðinni sem gengur undir nafninu
        tilvísunartalning. Koma þarf fram fastayrðing gagna og undir hvaða
        kringumstæðum minni er skilað.

        \item Lýsið einum kosti sem ruslasöfnunaraðferðin merkja og sópa hefur
        fram yfir afritunarsöfnun og einum kosti sem afritunarsöfnun hefur
        fram yfir merkja og sópa
    \end{enumerate}

    \bo{Spurning 13 möguleikar}
    \begin{enumerate}
        \item Sýnið BNF, EBNF, samhengisfrjálsa mállýsingu (CFG) eða málrit
        fyrir mál strengja yfir stafrófið $\{a, (, )\}$ þar sem svigar eru í jafnvægi.
        Dæmi um strengi í málinu
        \begin{verbatim}
            \epsilon (tómi strengurinn)
            a
            aaa 
            ()
            (((a)))
            a(aa(aa)a)aa
        \end{verbatim}
        Dæmi um strengi ekki í málinu.
        \begin{verbatim}
            (
            )
            )(
            ((a)
        \end{verbatim}
        \item Sýnið BNF, EBNF eða málrit fyrir mál segða yfir stafrófið $\{𝑥, +, (, )\}$.
        Svigar verða að vera í jafnvægi, + er tvíundaraðgerð og 𝑥 er
        breytunafn, sem er eina leyfða frumstæða segðin.
        Dæmi um strengi í málinu
        \begin{verbatim}
            x
            x+x+x
            (x)
            (((x)))
            x+(x+x+(x+x)+x)+x+x
        \end{verbatim}
        Dæmi um strengi ekki í málinu,
        \begin{verbatim}
            \epsilon (tómi strengurinn)
            (
            )
            2
            +x
            xx
            ((x)
            y
        \end{verbatim}

    \end{enumerate}

\end{document}